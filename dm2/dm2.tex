%!TeX TXS-program:compile = txs:///pdflatex/[--shell-escape]

\documentclass{article}
\usepackage[french]{babel}
\usepackage[a4paper,top=2cm,bottom=2cm,left=3cm,right=3cm,marginparwidth=1.75cm]{geometry}
\usepackage{amsmath}
\usepackage{amssymb}
\usepackage{graphicx}
\usepackage{listings}
\usepackage{color}
\usepackage{minted}
\usepackage{multirow}
\usemintedstyle{colorfull}


\title{DM Enveloppe convexe}
\author{Bartolomeo Ryan}

\begin{document}
\maketitle


\section{Fonction plus bas}
\begin{minted}{C}

    int plus_bas(int tab[][2],int n){
        int j=0;
        for(int i=1;i<n;i++){
            if(tab[i][1]<tab[j][1] || (tab[i][1]==tab[j][1] && tab[i][0]<tab[j][0] ))
            j = i;
        }
        return j;
    }

\end{minted}
\section{Correction}
On veut montrer la spécification :  \\ 
$E : T \subset \mathbb{Z}^2, $   \\
$S : T[j]=min_{\lessdot} T$ avec  $a \lessdot b  \Leftrightarrow e_y<f_y \lor (e_y=f_y \land e_x<f_x)$\\

On va prouver l'invariant : \\
$I(j,i): T[j]=min_{\lessdot} T_{[\![ 0,i ]\!]}$
\subsection{Initialisation}
Avant la boucle, $j=0$ et $T$ est un singleton.\\
$T[0]=min_{\lessdot} T_{[\![ 0,0 ]\!]}$ \\
donc $I(j,0)$ est vérifié.
\subsection{Iteration}
Montrons que pour tout rang $i$, si  $I(j,i-1)$ est vérifié, alors $I(j,i)$ l'est aussi.
\begin{itemize}
    \item Si $T[i]$ n'est pas plus petit par la relation d'ordre que $T[j]$, on ne modifie pas l'indice, donc $j$ rest l'indice du minimum
    \item Si au contraire, $T[i]$ est plus petit, alors $j=i$
\end{itemize}
Ainsi, par hypothèse d'itération, à la fin de chaque itération, $I(j,i) \equiv I(j,i+1)$ donc en sortie de boucle, $I(j,|T|)$ sera vérifié.

Ainsi, puisque la boucle termine et que \mintinline{C}{J} contient la valeur du point le plus bas, \textbf{la fonction est correcte}

\section{Exemples d'orientation}
\begin{center}
\begin{tabular}{ c c c c }
    i & j & k & Orientation \\ 
    0 & 3 & 4 & 1\\  
    8 & 9 & 10 & -1    
\end{tabular}
\end{center}
\section{Fonction orientation}
\begin{minted}{C}
int orient(int tab[][2],int i, int j, int k){
  int r=(tab[k][1]-tab[i][1])*(tab[j][0]-tab[i][0])-(tab[k][0]-tab[i][0])*(tab[j][1]-tab[i][1]);
  return (r ? r/abs(r): 0);
}
\end{minted}
\section{Relation d'ordre de $\preceq$}
$\preceq$ est une relation d'ordre car elle est réflexive, antisymétrique et transitive sur l'ensemble $P$.

\section{Fonction prochain point}
\begin{minted}{C}
int prochain_point(int tab[][2],int n,int i){
    int j=0;
    for(int k=0;k<n;k++){
        if(orient(tab,i,j,k)<=0 && k!=i)
            j = k;
    }
    return j;
}
\end{minted}
\section{Prochain point pour $i=10$}
\begin{tabular}{c c  c  c  c  c  c  c  c  c  c  c  c}    
j & 0 & 1 & 2 & 2 & 2 & 5 & 5 & 5 & 5 & 5 & 5 & 5 \\
k & 0 & 1 & 2 & 3 & 4 & 5 & 6 & 7 & 8 & 9 & 10 & 11 \\
\end{tabular}
\section{Fonction Enveloppe Convexe de Jarvis}
\begin{minted}{C}
int conv_jarvis(int tab[][2], int n, int *env){
    int i=plus_bas(tab,n);
    int ind=0;
    int p=i;
    do{
        env[ind++]=p;
        p=prochain_point(tab,n,p);
    }while(i!=p);
    return ind;
}
\end{minted}
\section{Temps d'éxécution}
Cette fonction prendrai dans l'ordre de $n+n*n$ opérations pour récuperer l'enveloppe dans le pire des cas, on estime donc sa complixé temporelle à $O(n^2)$

\section{Tri des éléments}
Parmi les algorithmes de tri répandus, on peut penser au tri fusion (ou \textit{merge sort}) qui a une compléxicité de $O(n ln(n))$
\section{Fonction Mise à jour de l'enveloppe supérieure}
\begin{minted}{C}
void maj_es(int tab[][2],stack es,int i){
    int t;
    int n;
    bool ordered=true;
    while(es->size >= 2 && ordered){
        t=pop(es);
        n=top(es);
        push(t,es);

        if(orient(tab,i,t,n)<0)
            pop(es);
        else
            ordered=false;
    }
    push(i,es);
}
\end{minted}
\section{Fonction Mise à jour de l'enveloppe inférieure}
\begin{minted}{C}
void maj_ei(int tab[][2],stack ei,int i){
    int t;
    int n;
    bool ordered=true;
    while(ei->size >= 2 && ordered){
        t=pop(ei);
        n=top(ei);
        push(t,ei);

        if(orient(tab,i,t,n)>0)
            pop(ei);
        else
            ordered=false;
    }
    push(i,ei);
}
\end{minted}
\section{Fonction Enveloppe convexe de Graham (variante de A.Andrew)}
\begin{minted}{C}
stack conv_graham(int tab[][2],int n){
    stack es=new_stack();
    stack ei=new_stack();
    for(int i=0;i<n;i++){
        maj_es(tab,es,i);
        maj_ei(tab,ei,i);
    }
    stack s=new_stack();
    pop(es);
    stack reversed_es=new_stack();
    for(int i=es->size;i--;)push(pop(es),reversed_es);
    for(int i=reversed_es->size;i--;)push(pop(reversed_es),s);
    for(int i=ei->size;i>1;i--)push(pop(ei),s);
    
    return si;
}
\end{minted}
\section{Analyse du temps d'éxécution}
Les fonctions de mises à jour prennent au pire des cas $\log n$ opérations pour être éxécutés.
Puisqu'on execute ces procédures $n$ fois, on peut estimer qu'on devra attendre l'éxécution de $2n\log n$ opérations, d'où la compléxcité de l'aglorithme est de l'ordre de $O(n\log n)$
\end{document}
