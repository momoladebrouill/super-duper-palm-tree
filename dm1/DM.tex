\hypertarget{devoir-maison-en-ocaml}{%
\section{Devoir maison en Ocaml}\label{devoir-maison-en-ocaml}}

\hypertarget{exercice-1}{%
\subsection{Exercice 1}\label{exercice-1}}

On peut implémenter de façon naive la suite de Fibonacci par sa
définition mathématique :

\begin{Shaded}
\begin{Highlighting}[]
\KeywordTok{let} \KeywordTok{rec}\NormalTok{ fib n =}
  \KeywordTok{match}\NormalTok{ n }\KeywordTok{with} 
  \DecValTok{0}\NormalTok{ | }\DecValTok{1}\NormalTok{ {-}\textgreater{} n}
\NormalTok{  | n {-}\textgreater{} fib (n}\DecValTok{{-}1}\NormalTok{) + fib (n}\DecValTok{{-}2}\NormalTok{)}
\end{Highlighting}
\end{Shaded}

Mais cet algorithme devient lent pour de grande valeurs. En effet, il a
une complexcité en O(n2) Une solution serait de faire un appel récursif
en branche simple :

\begin{Shaded}
\begin{Highlighting}[]
\KeywordTok{let}\NormalTok{ fib2 n =}
  \KeywordTok{let} \KeywordTok{rec}\NormalTok{ fib\_aux n =}
    \KeywordTok{match}\NormalTok{ n }\KeywordTok{with}
\NormalTok{    | }\DecValTok{0}\NormalTok{ {-}\textgreater{} (}\DecValTok{0}\NormalTok{,}\DecValTok{0}\NormalTok{)}
\NormalTok{    | }\DecValTok{1}\NormalTok{ {-}\textgreater{} (}\DecValTok{0}\NormalTok{,}\DecValTok{1}\NormalTok{)}
\NormalTok{    | n {-}\textgreater{} }\KeywordTok{let}\NormalTok{ a,b = fib\_aux (n}\DecValTok{{-}1}\NormalTok{) }\KeywordTok{in}\NormalTok{ (a+b,a)}
  \KeywordTok{in} 
    \KeywordTok{let}\NormalTok{ a,b = fib\_aux n }\KeywordTok{in}\NormalTok{ a}
\end{Highlighting}
\end{Shaded}

ici, on fait une récursivité terminale afin de miniser le coup en
mémoire et en temps : cet algorithme a une compléxicité de O(n)

\hypertarget{exercice-2}{%
\subsection{Exercice 2}\label{exercice-2}}

\begin{Shaded}
\begin{Highlighting}[]
\KeywordTok{let} \KeywordTok{rec}\NormalTok{ permute all =}
  \CommentTok{(*Liste des permutation des éléments*)}
  \KeywordTok{if}\NormalTok{ all = [] }\KeywordTok{then}\NormalTok{ [[]] }\KeywordTok{else}
  \KeywordTok{let}\NormalTok{ join l1 l2 = [l1;l2] }\KeywordTok{in} 
  \KeywordTok{let}\NormalTok{ remove elem l = }\DataTypeTok{List}\NormalTok{.filter ((\textless{}\textgreater{}) elem) l }\KeywordTok{in}
  \DataTypeTok{List}\NormalTok{.concat }\CommentTok{(*assemblage*)}
\NormalTok{    (}
    \DataTypeTok{List}\NormalTok{.map }\CommentTok{(*combinaisons*)}
\NormalTok{      (}\KeywordTok{fun}\NormalTok{ t {-}\textgreater{} }\DataTypeTok{List}\NormalTok{.map }
\NormalTok{        (}\DataTypeTok{List}\NormalTok{.cons t)}
\NormalTok{        (permute (remove t all)))}
\NormalTok{      all}
\NormalTok{  )}
\end{Highlighting}
\end{Shaded}

\hypertarget{exercice-3}{%
\subsection{Exercice 3}\label{exercice-3}}

\begin{Shaded}
\begin{Highlighting}[]
\KeywordTok{let}\NormalTok{ partition (n:}\DataTypeTok{int}\NormalTok{) (q:}\DataTypeTok{int}\NormalTok{) =}
    \CommentTok{(*}
\CommentTok{        On fera la liste l de membres de la somme et on compte le nombre de fois où cette somme tombe juste}
\CommentTok{        On suppose n\textgreater{}=q}
\CommentTok{    *)}
    \KeywordTok{if}\NormalTok{ n=q || q=}\DecValTok{1} \KeywordTok{then} \DecValTok{1} \KeywordTok{else} \CommentTok{(*cas de base*)}
    \KeywordTok{let}\NormalTok{ is\_good (l:}\DataTypeTok{int} \DataTypeTok{list}\NormalTok{) : }\DataTypeTok{int}\NormalTok{ = }\KeywordTok{if}\NormalTok{ (}\DataTypeTok{List}\NormalTok{.fold\_left ( + ) }\DecValTok{0}\NormalTok{ l) = n }\KeywordTok{then} \DecValTok{1} \KeywordTok{else} \DecValTok{0} \KeywordTok{in}
    
    \CommentTok{(*Modification des facteurs*)}
    \KeywordTok{let}\NormalTok{ (++|) a b = }\KeywordTok{if}\NormalTok{ a=n }\KeywordTok{then}\NormalTok{ b }\KeywordTok{else}\NormalTok{ a+}\DecValTok{1} \KeywordTok{in} 
    \KeywordTok{let} \KeywordTok{rec}\NormalTok{ increment (l:}\DataTypeTok{int} \DataTypeTok{list}\NormalTok{) (p:}\DataTypeTok{int}\NormalTok{): (}\DataTypeTok{int} \DataTypeTok{list}\NormalTok{) =}
        \KeywordTok{match}\NormalTok{ l }\KeywordTok{with}
\NormalTok{        | [] {-}\textgreater{} []}
\NormalTok{        | [x] {-}\textgreater{} [x ++| p]}
\NormalTok{        | t::q {-}\textgreater{} }\KeywordTok{let}\NormalTok{ t\textquotesingle{} = }\KeywordTok{if} \DataTypeTok{List}\NormalTok{.hd q \textgreater{}= n }\KeywordTok{then}\NormalTok{ t ++| p }\KeywordTok{else}\NormalTok{ t }\KeywordTok{in}
\NormalTok{             t\textquotesingle{}::increment q t\textquotesingle{}  }
    \KeywordTok{in}
    
    \CommentTok{(*Essais*)}
    \KeywordTok{let} \KeywordTok{rec}\NormalTok{ partion (l:}\DataTypeTok{int} \DataTypeTok{list}\NormalTok{) : }\DataTypeTok{int}\NormalTok{ =}
        \KeywordTok{if} \DataTypeTok{List}\NormalTok{.hd l \textgreater{}= n }\KeywordTok{then} \DecValTok{0} \CommentTok{(*si le premier membre a atteint n, on a fini*)}
        \KeywordTok{else}\NormalTok{ is\_good l + partion (increment l }\DecValTok{0}\NormalTok{) }\CommentTok{(*recursion des cas*)} 
    \KeywordTok{in}
\NormalTok{    partion (}\DataTypeTok{List}\NormalTok{.init q (}\KeywordTok{fun}\NormalTok{ \_{-}\textgreater{}}\DecValTok{1}\NormalTok{))}
\end{Highlighting}
\end{Shaded}

On obtient alors le tableau : Name Type Description Example ---- ----
----------- ------- \textbf{an-integer} \emph{Integer} A 64-bit Number
\texttt{12346} \textbar{} \textbf{0} \textbar{} 1 \textbar{} 0
\textbar{} 0 \textbar{} 0 \textbar{} 0 \textbar{} 0 \textbar{} 0
\textbar{} 0 \textbar{} 0 \textbar{} 0 \textbar{}
\textbar-------\textbar---\textbar---\textbar---\textbar---\textbar---\textbar---\textbar---\textbar---\textbar---\textbar---\textbar{}
\textbar{} \textbf{1} \textbar{} 1 \textbar{} 1 \textbar{} 0 \textbar{}
0 \textbar{} 0 \textbar{} 0 \textbar{} 0 \textbar{} 0 \textbar{} 0
\textbar{} 0 \textbar{} \textbar{} \textbf{2} \textbar{} 1 \textbar{} 1
\textbar{} 1 \textbar{} 0 \textbar{} 0 \textbar{} 0 \textbar{} 0
\textbar{} 0 \textbar{} 0 \textbar{} 0 \textbar{} \textbar{} \textbf{3}
\textbar{} 1 \textbar{} 2 \textbar{} 1 \textbar{} 1 \textbar{} 0
\textbar{} 0 \textbar{} 0 \textbar{} 0 \textbar{} 0 \textbar{} 0
\textbar{} \textbar{} \textbf{4} \textbar{} 1 \textbar{} 2 \textbar{} 2
\textbar{} 1 \textbar{} 1 \textbar{} 0 \textbar{} 0 \textbar{} 0
\textbar{} 0 \textbar{} 0 \textbar{} \textbar{} \textbf{5} \textbar{} 1
\textbar{} 3 \textbar{} 3 \textbar{} 2 \textbar{} 1 \textbar{} 1
\textbar{} 0 \textbar{} 0 \textbar{} 0 \textbar{} 0 \textbar{}
\textbar{} \textbf{6} \textbar{} 1 \textbar{} 3 \textbar{} 4 \textbar{}
3 \textbar{} 2 \textbar{} 1 \textbar{} 1 \textbar{} 0 \textbar{} 0
\textbar{} 0 \textbar{} \textbar{} \textbf{7} \textbar{} 1 \textbar{} 4
\textbar{} 5 \textbar{} 5 \textbar{} 3 \textbar{} 2 \textbar{} 1
\textbar{} 1 \textbar{} 0 \textbar{} 0 \textbar{} \textbar{} \textbf{8}
\textbar{} 1 \textbar{} 4 \textbar{} 7 \textbar{} 6 \textbar{} 5
\textbar{} 3 \textbar{} 2 \textbar{} 1 \textbar{} 1 \textbar{} 0
\textbar{} \textbar{} \textbf{9} \textbar{} 1 \textbar{} 5 \textbar{} 8
\textbar{} 9 \textbar{} 7 \textbar{} 5 \textbar{} 3 \textbar{} 2
\textbar{} 1 \textbar{} 1 \textbar{}
